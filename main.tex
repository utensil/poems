\documentclass{article}
\usepackage[utf8]{inputenc}
%漢字自動注音
\usepackage{xpinyin}
%漢字注音設置
\xpinyinsetup{ratio = .4,font =\ubufont,multiple={\color{black}}}
%制作诗歌显示环境
\usepackage{tcolorbox}
\usepackage{fontspec}

% \setmainfont{[fzfss.ttf]}
% \setsansfont{[fzfss.ttf]} 
% \setmonofont{[fzfss.ttf]}
% \setCJKmainfont{[fzfss.ttf]}
\setCJKmainfont{FandolFang}
%设置中文正体字体,BoldFont设置粗体和斜体样式对应的字体
\setCJKsansfont{AR PL UKai CN}%设置无衬线样式对应字体
% \setCJKmonofont{[fzfss.ttf]} %设置有衬线样式对应字体

%邊註交互引用
\NewDocumentCommand\pref{m}
{\marginpar{\ding{43}\HGB\ubufontc\hyperlink{#1}#1(\pageref{#1})}}
%正文交互引用
\NewDocumentCommand\lref{m}
{{\HGB\ubufontc\hyperlink{#1}#1(\pageref{#1})}}

\newenvironment{poem}[3] %code={\doublespacing},
{\begin{tcolorbox}[left=1mm,right=1mm,top=3mm,bottom=2mm,enhanced,breakable,pad at break*=.5mm,title=\begin{pinyinscope}\begin{center}\linespread{1.4}\selectfont #2\end{center}\end{pinyinscope},
		colframe=blue!50!black,colback=blue!10!white,colbacktitle=blue!5!myred!10!white,
		fonttitle=\huge\HGBM,coltitle=black,attach boxed title to top center=
		{yshift=-0.25mm-\tcboxedtitleheight/2,yshifttext=2mm-\tcboxedtitleheight/2},
		boxed title style={boxrule=0.5mm,
			frame code={ \path[tcb fill frame] ([xshift=-4mm]frame.west) -- (frame.north west) -- (frame.north east) -- ([xshift=4mm]frame.east) -- (frame.south east) -- (frame.south west) -- cycle; },
			interior code={ \path[tcb fill interior] ([xshift=-2mm]interior.west) -- (interior.north west) -- (interior.north east) -- ([xshift=2mm]interior.east) -- (interior.south east) -- (interior.south west) -- cycle;} }]
      \begin{pinyinscope}
      
      \begin{center}
      
      \LARGE\linespread{1.4}\selectfont #3  %\HGBUM\ZSKS
	}
{\end{center}\end{pinyinscope}\end{tcolorbox}}

\title{冶文斋诗选}
\author{心皿}

\begin{document}
\date{}
\maketitle

\section{序:在日常里写旧体诗的一点体会}

受到语文教育古诗词熏陶的每个人,或许都尝试过用旧体诗来书写,至少也尝试过改编原诗,或者写些打油诗,这个阶段没有太高门槛。但若要继续往前,文化积淀、诗词格律形成了技术门槛,鉴赏水平、敏锐的心灵、意象、诗意、思想等,形成了软性门槛。所以,在日常中,诗更多是用来吟诵、赏析、引用,如果引用的时候贴合情境,那就已经是很不错的运用了。也有爱好者形成的写诗的群体,但对于彼此写诗的水平,看法可能很不一致,局面可能是,作者孤芳自赏,圈中人不以为然,而我们只能在圈外不明觉厉。

旧体诗,在日常里,只能处于这般只读的状态,无法拥有交互和书写的体验,是极其可惜的。是否可以从一个比较低的起点开始,尝试融化部分门槛,让旧体诗成为自己可以在日常中使用的一种文体,表达自己真情实感呢?我发现是可以的。

可是,从根子上说,那么多文体可供选择,为什么要写旧体诗呢?

在我还没有习惯写旧体诗的时候,我曾写道:``我觉得诗意和诗的形式是存在的。但我觉得诗和诗人是不存在的。诗不能独立作为写作对象而存在,它,正如词是诗余一样,是文余。诗意也是意义的余,一种副产品。它是意外而美丽的凝结,它的纤巧无法刻意。它是那个在做着想着其它事情的人,无意中给后人留下的词句。所以我觉得,很多单独成诗的,反而不是诗。洒洒行文之间,突然读到的那一两句惊心的句子,却常常是更美的诗。它们正如嵌在山岩里的玉璞,是诗的原初形态。''所以那时我在日常里,更喜欢的是在行文中自然带上诗意,而不是专门以诗作为一种文体。

后来,在后微博和朋友圈的时代,我会特别喜欢用旧体诗来作为一种表达方式。整体来看,它在字数规则上比较简单固定,不像长短不一的词,而无论是内容上和形式上,细节处都有很多打磨的空间。为了在短小的篇幅中表达丰富的意思,它有一种追求言简意赅、字字珠玑的精致,是非常浓缩的精华,事后读来,有点就像《三体》里质子大小的智子的低维展开一样,意蕴无垠、回味无穷。

因为我的工作很忙,并没有时间像学生时代那样大段地去记录发生的故事和所思所想,我写诗更多地是作为一种记录生活与调节心态的方式:那些一时无法忘怀的事情与感受,如果我把它凝结到一首诗中了,就仿佛可以由它来替我铭记和承载那些感受,而我可以回到日常中,继续扮演我的角色,只在后来某一天,在可以有时间岛屿的奢侈时,回到诗前,重新展开那段经历和那片心情。

诗作的水平不要紧。开始写时,总不能去和那些经过历史大浪淘沙留下来的经典去比较。《香菱学诗》的故事里讲的写诗的道理,都是很好的,但香菱的诗到第三首便有那般水平,是不科学的,甚至要达到第一首那样成功地堆砌前人词藻,也是殊为不易的过程。写诗,和写任何文字一样,更多地是一种只与作者自身相关的体验,将作者的见闻、感受、思想,凝结到诗中,自己在时过境迁之后读来,还能唤起当时的感受,那便已经是成功。

人类的悲欢,不说不相通,至少在大多数情况下,个人的体验都是非常独有,而难以唤起其他人的共情的,更别提像经典一样,跨越时空感染后人。这便如唱歌,无论基础如何,自己K歌,总是可以唱好、尽兴的,但若要在旁者听来好听,甚至成为歌星,那完全是另外一回事。舞蹈也是每个人都可以使用的一种舒展变换自己肢体的艺术,只要自己跳着舒服就好,但若要在旁人眼里形成美感,则需要严格的基础功训练。

所以,要开始,就要先以习作的心态来写诗,只要能够有效地凝结了那一刻的时空与自己,在词句、立意任一维度稍有可取之处,便很好。

当然,如果自己那一刻根本没有和诗相关的灵感,或者要表达的内容,根本不适合诗,或者不适合旧体诗,那就要果断选择其他的文体,别来招惹诗。也不要强行穿越到古人的时空里,以古人、别人的主题写诗。可以尝试命题写诗,但命题必须从自己确实存在的体验中来找,近的可以是旅游时见到的景观,远的也可以是刚看了一本书或者一部剧,有所感的故事。

旧体诗作为一种文体,要求作者拥有敏锐的观察和分析周遭的能力,对内心的思想和情感也要非常敏感,然后需要具备非常感性和具象的想象力,还需要能够有一定的强迫症或者秩序感,以及盈余的智力,来去探索诗的格律要求的音韵美和形式美。这些特质是天生的,但我相信每个人都或多或少具备。

某种意义上,格律是一个普通人写旧体诗最大的心理障碍。首先要学会自身与之相处的方式,才能真正动笔去写旧体诗。作为一个不喜欢繁复的记忆负担,也不喜欢为规则所约束的人,我在相当长的时间里,都无法走近格律。但我隐约知道,它是要去趋近的正道。我曾经借阅或者买过许多讲诗词格律的书,都很好,很严谨,很有文化,我却始终不得其法。说不清是不是只是第七个饼,最终在遇见《格律诗新探(唐人引以为豪的当时体)》》之后,突然有了感觉。也是在那之后,我才突然觉得写旧体诗的世界对我不再关闭,而成为了可以交互的对象,也开始留下了今天看来还算诗作的尝试。

我找到的感觉是,既然是习作的心态,那么在格律方面,其实可以有很大弹性。``不要以律害意''首先可以确立是正确的最高原则,虽然它经常被用作不去好好学习格律的借口。很多时候,个人功力没到,想表达的意境,就是没法找到合乎音韵、平仄的字词来去表达,如果的确是这种情况,这首诗,只能炼到这儿了,要承认,把好的、已经完成的部分,保留下来。当然,格律是一定要学习掌握的,写诗时也应该瞄准最严谨的近体诗的标准,在用韵方面,首先瞄准平水韵,然后再根据力所能及,酌情降级到普通话的韵,甚至有时候意象、词句觉得好,放弃押韵也是可以的。最后这种情况,可能很有争议,但如果去翻阅一些成名诗人的全集,就会偶尔发现这样的例子,所以不必执着。

弹性并不意味着降低自我要求。诗意犹在的时候,可以动用盈余的智力,去尝试贴合格律。格律本身是对音韵美、节奏美的客观规律的总结提炼,是一个应该趋近的标准。本质上,是因为前人已经有了很多诗作,在宽松的要求下已经有了很多佳作,所以,近体诗意义下的格律才诞生出来,作为专业人士进一步追求卓越的指引。它是一个关于形式美的文字游戏,是一个求解方程的过程,它蕴含的是一种解谜的乐趣。如果在立意、意象等方面已经恰好有所得,如果可能,为何不进一步打磨其形式,说不定就可以再美一点?当然,艺术的追求是一个不确定是否收敛的过程,如果力竭,那么就要安于所得。诗,作为上天的赠与,它的美丽总是有限的。在这首诗里未达到的,还可以有下一首的机会。

我非常早期的一首诗作,就是连押韵都没顾及:

\begin{quote}
唇间春意露,身后臂膀环。安居甜蜜意,携手浪迹心。
\end{quote}

创作背景是:

\begin{quote}
世间男女偶遇相恋后,虽情意绵绵,心中所憧憬的,却是不一样的幸福。小龙女要的是终老古墓,杨过要的是携手浪迹天涯。女性常以家庭为终极价值、归宿,男性以家庭为事业生涯的基石、港湾。这种爱恋的引力和心意的相背,是构成了长期相处的张力。于花市看见一对甜蜜情侣,心有感而作。
\end{quote}

虽然音韵平仄皆是完全离谱,此作有其他方面的可取之处。形式上,前两句是4字名词+1字动词的结构,后两句是4字形容词+1字名词的结构,分别都算对仗整齐。前两句写具体看到的情形,后两句升华写感受,一三写女性,二四写男性,结构上对应。个人理解,这是比较典型的,早期尝试可以达到的形式美状态。用词上,不会过于日常口语化,也并没有怎么文雅,其语感大概在放在诗里不至于不伦不类,或者有点像成语的程度。

这首诗,有画面感,有情感,有思考,有一定的形式结构之美。对于初写诗者,已经照顾到了这么多方面,夫复何求?

尤其是在画面感与思想之间的平衡上,是不容易的。有的时候,全诗都写成了思想,有的时候,全诗都写成了画面。

前者的早期例子是(创作背景是工作上有很大困难要去克服,才能实现承诺了要做到的上级的要求):

\begin{quote}
百挫得来倍珍惜, 阻滞关头意须决。 于无措中求巧解, 为矢寻的践其约。
\end{quote}

全诗就用了一个箭的形象,其他全是思想。

后者的早期例子是(创作背景是回到母校校园):

\begin{quote}
徐风草荷香, 迟波逸思漾。 叶雨洒苔阶, 抚卷掩幽肠。
\end{quote}

全诗都是写景,最后有一个幽肠的情感,但与景色并无关联,只是的确是我在校园里想起了很多过去的事情产生的情感。

除了形象和抽象之间的平衡之外,也需要注意``乐而不淫,哀而不伤''。我有的诗写于工作压力很大的时候,整体风格就极度阴暗抑郁,比如:

\begin{quote}
苦营杂役形枯槁, 重压潜流心焦颓。 鸡血干货有时尽, 执念愿景转成灰。
\end{quote}

这几个例子,我们不去讨论它们是否是佳作(事实上,我的诗作里勉强能竞争佳作的都偏后期,也都更合格律,或许未来才能登大雅之堂,但前期的时空、感受、稚嫩而认真的尝试,都一去不复返了,所以在本文的语境中,反而更有参考意义),但它们都起到了那一刻我希望它们起到的作用:写出来,忘记它,未来回味。

从这几个例子的用词也可以看出,我不是很追求白居易那``老妪能解''式的明白易懂。我更追求把意思用字词准确地捕捉和刻画出来,只有掌握了那一刻体验真相的我自己能判断用词的准确性,或者像《红楼梦》那样用曲笔藏起来,我只蜷缩在层层外壳之内。诗作读起来,需要一个字一个字地理解,辨识其中的背后深意,可以是需要另外解释才能够体会。我相信写诗是一种编码的过程,而解码是有门槛的。

我偶尔会在诗中用上生僻字或者比较生僻的连绵词。这些字或者词,看到时我们隐约知道是什么意思(因为在别的语境下似乎见过),写的时候你会模糊地想起想用这个词,不一定回忆得起它的完整发音,也不一定记得它的完整字形,所以直接用不出来,但凭着那种感觉经过一定的搜索,能够找到它。当它合适这首诗时,我想起了它,找到了它,再把这块宝石镶嵌在了我的诗中,从此它便是我的,而不再是故纸堆里已死的存在和今人记忆的负担。

可是,如果如此晦涩,岂不是完全成为了无法阅读的孤独的密文?我相信诗是可以有很多层的,虽然深入解读下去,诗的细节能藏住我的很多想法,但乍读起来能够像看一幅画的缩略图一样,领略到画整体的意境和情感。我心目中追求的诗作的极致是:看起来是一个样,稍微解读一下,就到达了对表象的否定,但再深入到内核,却又与表象相协调一致,形成否定之否定,或者诗吃起来一开始是一种味道,但每舔开一层,都有新的五味。

这便是一种意义的编织,它不是一种刻意,而是因为诗人在那一刻经历的认知与情感,本身就是没有一致性、没有逻辑性的意识流,是一种丰富多层次的体验,所以,也可以把它注入到诗的多层、多向的编织结构中。正如我在《雕琢》里面写道:``雕琢最是不经意间的投入,不若刻舟求剑的执着。世间的事情,若作为机械的任务,难得发指,若意起雕琢起来,完成时只觉得自然而毫无阻滞,亦察觉不出落笔之繁复。''

这个也反向对诗的句式提出了要求。我不喜欢过于口语化,像一句话一样的直抒胸臆的句式。这样的句式过于浪费字数,一句诗就只能表达一种意思。这个时候,电影里蒙太奇的手法是非常好的。我们不需要点出所有的主语、宾语,也不需要直接地把事物之间的细节联系用动词表达出来。我们只需要把每件事物或者事情用直接或隐喻的方式刻画或代表,然后把它们摆在一起,布置成一片天地,让人类那渴求建模的大脑来完成诠释。如果我们真的要去使用动词,也应该是针对这个画面整体的感受,主语和宾语都不再是这幅画中的某个组成部分,而是整张画。这个手法背后,也是诗对我们的想象力的要求,某种意义上,想象力正是对于潜在的可能联系的发现。

不重字的规则,我一直都特别在意,每一首都坚持。全诗中用字不能重复,诗中原则上不能出现标题中有的字。对于要描写的事物和情感,我们不要直接呼唤它的名字。诗中处处出现它的影子,但就是不要直接出现它的本体。意思可以重复,形成结构性的呼应,但其形态必须变幻。新诗、长诗、歌词或者《诗经》,更适合咏叹式的冗余重复,但旧体诗在绝大多数情况下不适合。唯一允许的例外,是使用叠字的连绵词,或者顶针结构,但适合的场景极为罕见,也可能会冲淡诗的浓度。

``诗言志。''``文章憎命达。''工作中的我们,已经远离了少年时代的理想、情结、或者拥有其他名字的初心,但也从未忘却。在工作中,基于现实的材料,在整体的形势下,我们也偶有愿景与豪情。在压力巨大的工作中,我们会遇到许多挫折。在职业生涯的发展中,我们会遇到困顿与曲折,也会面临需要勇气的抉择。在情感或家庭生活中,我们有天伦之乐,有各种突发变故,也有许多心事。在聚餐、旅游、K歌等娱乐活动中,我们拥有过刹那的放开的自我。这一切,都可以酝酿成诗,这些体味,都可以入诗。在相当长的时间里,我都是相隔数月,才凝成一首诗,标志性地记录一段生涯。或者偶尔,因为在关键节点上,我的内在挣扎或者反思很多,一件事产生多个截然不同的想法,会连续有几首不同的诗。又或许,一次旅游,就促使诗兴大发。每到一处景观,产生感受,用好赶路的独处时间,便是一首诗。甚至一处景观,因为不同角度产生的感受迥异,便写出多首诗。一首诗的四句,可以正好一一对应一起发在朋友圈的四张图。有时可能从某句聊天触发诗兴,到一挥而就,再修改到合乎格律,只用了十几分钟。每首诗,我要写的,既是那一刻,是那一个具体的事情或者事物,又会不由自主地折射出自己身处的整个局,或是结晶了跨越半年或者数年这样的时间尺度的零碎心绪。

这便是诗的选题。旧体诗,的确是一种很有意思、很特殊的载体,它的确也能承载一个现代人,一个从事互联网技术工作的人的生活与思想。当然,它只是无数的载体之一。公域的微博或者私域的朋友圈,可以有很多丰富的形式:在媒体上,可以有照片、视频、音乐;在文本上,可以有短句,可以有散文,可以有新诗,可以有旧体诗,可以有英语或者法语,还可以有UnicodeArt或者Emoji。在这个光怪陆离的世界中,旧体诗可以有它的一个位置,一个像茶一样的位置,成为我们日常表达的一部分,观照自我和世界。

在现代写旧体诗,自然有一些新的方法,实际上也不可避免会用到一定的科技。科技的引入,并不会削弱诗本身的古典意味。毕竟科技只是工具,人才是掌握工具的主体。就算人类进入了星际大航海或者银河帝国的时代,那些情怀依然与古人没有什么不同,一样是一个有着自身有限性,但在思想和情感上却可以无垠的有情生命体,面对这个充满未知和不确定性的世界,所发出的面向永恒时间的短短心声。

在这个诗艺没落的时代,一个普通人,通常不会对古籍耳熟能详,不会对韵典烂熟于心,也不会在肌肉记忆中刻印那么多格律的规则。一个手机App能起到很大的帮助,消融这里的门槛。以前我用过``诗词吾爱'',后来主要是用``搜韵''。

诗兴来的时候,我会用纸笔先涂涂画画,纸上会散落半句、字词、意象等混沌的脑图,渐渐可以用行书写为一首完整结构的诗,又涂涂改改,在自有语感的指引下,寻找更贴合和表达和字词。到这里,完成了自身能力支撑的前半截创作。然后把诗打字到手机里,复制到搜韵的``创作''功能里,让它校验用韵、平仄,并且给出修改建议。

``搜韵''校验和建议的功能并不是很完善,但可以适应和善用:

1、它能识别出当前的用韵,但不能直接点击跳到对应的韵部,韵部也无法搜索,需要单独去肉眼找韵部;

2、对于多音字的平仄,它给出分别的发音和词义,让自校,但没有明确提示应该平还是应该仄,需要手工输入一个明确的平仄字来确认;

3、对于出韵、或者不合平仄的字,根据前后文,它能提示一些替换的组合,但通常因为意义方面的原因不能使用,不过,作为一个联想的种子是很好的;

4、韵典中字的排序,没有逻辑性,没有按照古诗词中出现的频率来,导致常用字和生僻字混杂在一起,找合适的字的体验不够流畅,有点打断心流;

5、缺乏流畅地从校验结果中的某个字词,跳转到对应的字典,再回到当前校验结果中的自然操作流,需要把``创作''和作为字典的``发现''当作两个独立的功能来用;

6、对于失粘、孤平、拗救等更复杂的情况,提示很有限,应该还可以做多一些。

对技术的使用,永远不能是机械化的。字词韵典,撒播的是灵感的随机种子,我希望找寻的那一粟,依然是自己记忆和积淀的海洋里有的,不会强行去使用我自己的积淀中毫无触及的典故或生僻字。格律规则,牵引的是表达的探索,我希望找寻的,是比第一时间想到的字词、句式更好的表达,但如果原初的喷涌就是最美的形态,我们就要回到原初。

这个过程中,个人总结,一首诗的创作,可以分为八个步骤:

1、灵感入态

每次写诗前,都是需要突然有灵感,带我进入状态。这时会有一种模糊的诗意要去捕捉。这样的灵感是一种不可控的心血来潮。

灵感本身很有可能就是半句诗,它一下子奠定了全诗的基调和句式,它是一个种子和起点,后面的一切都是围绕着它衍生出来的。

2、所思立意

灵感的产生都是有基础的,正所谓``日有所思,夜有所梦'',灵感的产生是因为前面已经有了一系列的情感和思想在莫名游动。需要找到这些思想,围绕它们确定若干个可能的立意。

3、见闻取象

每首诗都需要一些具体的画面,或者是一个有隐喻的意象。它们都必须从自己的所见所闻中来。快速回忆这几天见过的事物、风景、人物等等,把有价值的具体形象,铺展在眼前。这一步和上一步是并行进行的,而且往往交缠在一起,不断产生新的想法。

4、定式构形

这一步是从以往读过的诗词中,找一些固定的句式、对仗结构,把全诗的基本形状确定下来。五言就要确定比如是2字主1字谓2字宾,还是4+1字的主谓结构或者形动结构,比较可数。七言较为复杂,主要涉及词性在前4后3之间的分布,这个过程中,要优先确认动词的位置。

5、 押韵定调

定式构形更多地是针对一句之内,或者两句对仗的句子之间。而押韵定调,涉及到要从前面几个环节确定的潜在第二句的尾字的韵部,并且构思第四、六、八句具体选择该韵部下的哪个韵字。一旦这个韵被确认了,二、四、六、八句就可以串起来考虑整体的呼应和搭配了。

韵部的选择直接限制了每句意义收尾的空间,很多时候一些好的意象或者词句,会被韵部所不容。这个时候若想保留原意,可以调整四、六、八句内的语序,或者修改第二句的韵部。很多时候,我也接纳了韵部牵引出来的另外的立意方向。

6、对仗节句

随着第4步定式构形确定了部分单句内部的结构,押韵定调确定了全诗的呼应方式。这里就要用对仗来确定相邻两句之间的结构,来节制相应的句子。

不过,不能每两句都是对仗,形式上会显得单调。有时我会根据表达的需要,在结构基本对仗的基础上,做出反对仗的调整,比如两句的动词落在不同的位置上,以使得结构灵活,错落有致。而如果两句整体形成一个长句式的话,两句之间更是不需要对仗,而是可能形成递进、转折等修辞关系。

7、 平仄酌字

基本上,经过了前面几环,到这里,一首诗的雏形就出来了,但还需要就格律上进行打磨,这个过程,因为为了平仄,往往需要重新斟酌字词,说不定在这个牵引的过程中,就会找到更合适表达意思的字。

8、通读取舍

经历过了要押韵、要符合平仄的过程后,基本上全诗和最开始写的诗已经很不一样了。字眼、句式甚至整体思想,都可能受到了影响,甚至扼杀整首诗最关键的创意。这个时候,就要对比这首诗从没怎么改,到按格律基本全部改完整个过程中产生的各个诗的文本,然后取舍选择最好的字眼和整体意思。有时候,要忍心从完全贴合格律的诗态,退回去一步,取更能表达意思的字词。

以上基本已经分享完我的体会。全文可以写成更有逻辑性的金字塔结构,但我故意要写成内容与形式、动机与技巧交缠在一起的形态,更符合本文所处于的诗的创作意境。作为结语,我还想进一步在平行宇宙的背景下,谈谈诗的创作感。这种创作感,正是为什么即使是习作的质量,坚持把诗写出来,依然是一件值得做的事情,而不能只停留在复读前人。

创造力是发散与收敛的一种平衡。我们需要想象力的发散来找寻跳跃的灵感,又需要专业度的收敛来结构性地组织。这个过程,就像量子物理中波函数对于概率的衍生,以及最后观测时结果的坍缩。我们必须取舍字词与意思,可一瓢之外,还有弱水三千。当一首诗凝结在我们的笔尖,它的美依然是发散到各个平行宇宙中去的。在这个宇宙里的我们,得到了这一首它,也依然可以相信,包含被舍弃了的字词的变种还在其他的平行宇宙里,继续演化,以另一个终极形态盛开。

写出它的我们,在别的平行宇宙,或许可以有更深厚的文化底蕴,更高的写诗技巧,但在这一时空,我们作为宇宙的褶皱,折叠出来的,便只是这一只纸鹤。或许生涩,或许粗糙,或许有错讹,但重要的是,这是我们的造物,它不是宇宙最初的法则中直接蕴含的,而是宇宙演化所创造的我们所创作的,全新的、唯一的造物。无数个平行宇宙中的我们,走了半途,走入歧路,或是止步不前,我们才在这一个宇宙中,采撷、拾掇字词,到达了它。

\section{职业生涯}

\begin{poem}{}{时机}
风云际会无好手,

千里马来非佳期。

熙熙攘攘不得志,

形形色色难如一。
\end{poem}

\begin{poem}{}{补天}
每每功亏谁知苦,

夙愿得偿笑颜开。

累年残疾一朝愈,

流失用户还复来。
\end{poem}

\begin{poem}{}{心旗}
百挫得来倍珍惜,

阻滞关头意须决。

于无措中求巧解,

为矢寻的践其约。
\end{poem}

\begin{poem}{}{屡战}
筹营后方不甘寂,

杂中炼精视非途。

弃者岂能垮我辈,

再塑箫规重上路。
\end{poem}

\begin{poem}{}{断舍}
远景渐朗继有人,

重燃狼性断舍离。

案头宗卷恍隔世,

再铺新纸染心血。
\end{poem}

\begin{poem}{}{统帅}
倾采思量行果决,

步子错落渐有致。

登高远望酿山河,

绳心养性融诸识。
\end{poem}

\begin{poem}{}{雨澜}
羡欲摇曳川蚀岩,

志安心谷雨汇洼。

念顽意涩以真灼,

锤炼研磨候缘期。
\end{poem}

\begin{poem}{}{整装}
卸壳拆戏剖心柳,

意寒行豫沐暖颜。

接骨缝漏重举伞,

铸诚为翼越渊堑。
\end{poem}

\begin{poem}{}{战时}
弛弓引势砺筋骨,

逐猎令出箭夺的。

就材起灶拼白刃,

莫待兵成恨狼藉。
\end{poem}

\begin{poem}{}{泳夜}
一章功成万牍耽,

载愿千钧辙痕浅。

身投静波辉影间,

念出屏囚另开天。
\end{poem}

按:是夜,与妻女夜泳嬉戏,少顷,妻女还家,独自泳于夜色之中。近日案牍之上,多有事倍功半、愿实背离、罔顾错憾、悬空未决之事,未达使命之感挥之不去,心念中轴转无隙。累时,往往流连美剧,或沉浸手游,不过仍是屏幕的囚徒,以一时之娱回复能量,并未真正休憩。直至回归泳池,目之所及皆是淡蓝辉影,千头万绪,方归于静谧,是以凝成此诗。

\begin{poem}{}{拭剑}
缩岛沉舟谐何妥,

应付终托痂涩喜。

誓绝因庸疑误己,

择伙践志斩荆棘。
\end{poem}

\begin{poem}{}{型势}
宝剑初锋无当者,

细琢日用千面同。

鞘藏温养愈敛涵,

逢境但出破长空。
\end{poem}

\begin{poem}{}{伙伴}
成败中道凝风云,

是非成空伙相觑。

执事当惜眼前伴,

共历契心战相许。
\end{poem}

\begin{poem}{}{投融}
盘综境陌张景怀,

斗转晨曦筹措形。

临渊望岸纵身跃,

漠荒岭峻向北星。
\end{poem}

\begin{poem}{}{直前}
一叶入秋既别枝,

凭风起舞借扬时。

狡兔恋窟伤末途,

勇往兼济未可知。
\end{poem}

\begin{poem}{}{信使}
迟翔青鸟赴淤途,

知候犹诺愿履辛。

纷扰投射居中炼,

蛰坐辗转舒长心。
\end{poem}

\begin{poem}{}{闯潭}
起航远望轻风雨,

险鲨恶浪守机舱。

展振蝶翅牵调顺,

云泥扑朔引天光。
\end{poem}

\section{压力}

压力大时所写,用以倾泻心中负能量,整体偏负面,但起到了心理调节的作用。

\begin{poem}{}{劳疾}
苦营杂役形枯槁,

重压潜流心焦颓。

鸡血干货有时尽,

执念愿景转成灰。
\end{poem}

\begin{poem}{}{多艰}
白日连轴转,

夜晚噩梦缠。

拿起庶几成,

放下乱成团。
\end{poem}

\begin{poem}{}{还家}
连轴苦战为年安,

纷至迭来案无垠。

身在樊笼心企渴,

一朝放飞疚难平。
\end{poem}

\begin{poem}{}{肩钧}
扶犁耕新野,

授将镇旧疆。

奔走拨千斤,

失察辛莫赎。
\end{poem}

\begin{poem}{}{未竟}
殚精竭虑周不全,

连迭赶忙未如期。

年年规划竟半功,

岁岁值守起新波。
\end{poem}

\begin{poem}{}{低效}
勉力营馨难兑诺,

案牍烦忧蚀图景。

熙攘无处解孤渴,

思华星火成烟烬。
\end{poem}

\begin{poem}{}{岩浆}
时偿积欠缺填勉,

凝途失涩兴阑珊。

奔流方遒黯如没,

趣中汲静思里安。
\end{poem}

\begin{poem}{}{望空}
立身荒谬耕心意,

倾注殷切抹能轻。

因果涟漪噬回还,

于墟棘中缮草亭。
\end{poem}

\begin{poem}{}{掌中}
势成相左湮持庸,

幽阵摊推守驰驱。

颠簸诡谲栖疏惬,

井井施条焕遗墟。
\end{poem}

\begin{poem}{}{齿轮}
念执临穴洑水行,

触域及高知位险。

十方错切拢散盘,

千辛半阙耕寥田。
\end{poem}

\begin{poem}{}{将息}
施张百挫知危困,

润物开锋竭所及。

且听雀声荐鸩媒,

收帆稳舵抚蓑笠。
\end{poem}

\section{爱情与家庭}

\begin{poem}{}{心印}
盈盈倩影跃,

嫣嫣美目盼。

各归意伶俜,

寤寐思身畔。
\end{poem}

\begin{poem}{}{心意}
唇间春意露,

身后臂膀环。

安居甜蜜意,

携手浪迹心。
\end{poem}

世间男女偶遇相恋后,虽情意绵绵,心中所憧憬的,却是不一样的幸福。小龙女要的是终老古墓,杨过要的是携手浪迹天涯。女性常以家庭为终极价值、归宿,男性以家庭为事业生涯的基石、港湾。这种爱恋的引力和心意的相背,是构成了长期相处的张力。于花市看见一对甜蜜情侣,心有感而作。

\begin{poem}{}{喜临}
同事儿女接连诞,

意犹忐忑羡天伦。

忽如一夜豆发芽,

山花烂漫面春风。
\end{poem}

\begin{poem}{}{十月}
纸迹浅浅映愿深,孕吐翻腾喜含酸。

唐筛排查换心宽,四维彩超初见形。

萌童海报环四壁,典乐诗谣开鸿蒙。

腹隆曲美留存照,五谷杂粮控血糖。

腰背腿疼卧难安,胎动如悸渐踢蹬。

脐绕不再位已正,入盆待命盼宫缩。
\end{poem}

\begin{poem}{}{疹热}
反侧眠未安,
煎熬口难言。

楚泪湿枕席,
内焚红耳弯。

父母千重忧,
诊者一言宽。

轻身脱病沼,
复享天伦欢。
\end{poem}

\section{哲思}

\begin{poem}{}{孤僧}
花市灯华人如潮,

我缀其中如孤僧。

皮色红尘愿尽历,

只求得透空中道。
\end{poem}

春节花市所作。

前两句是闹中孤寂,华景空心。

后两句中,色相狭义指可被感知的世界,广义指围绕主体的符号秩序,是心相不是实相。皮相则是借指实相。存在是不可能的,对存在的意识更是不可能的,然而我们却确实活在双重不可能中。不可能的存在唯一诱人的是它的之所以可能,亦即道。没有道,一切便都是空,都是色。

\begin{poem}{}{花泥}
世间本来无有我,

何必以我求忘我。

悲喜触识着此在,

百年之后化花泥。
\end{poem}

见“求忘我”语所作。

在贪著妙触的基础上,又增加了“识”,代表着求知欲,代表着知识在“我”这里的融合。花泥是有序生命体的消散,也是新生生命体的养分。

所以,虽然“我”是空的,却融入了更大的生生不息的空。这样的人生图景,美丽而意足。

\begin{poem}{}{衣悟}
德责勉尽咎愧仍,

两全难能何缚己。

沧粟漂萍负块垒,

不如一骑驰心野。
\end{poem}

\section{心情与事件}

\begin{poem}{}{相寻}
走马前边行,

流连人没漫。

众里茫相寻,

一枝蝴蝶颤。
\end{poem}

花市人多走散寻回所作。取“众里寻他千百度,蓦然回首,那人却在,灯火阑珊处”的意境。

\begin{poem}{}{年心}
冬寒指日还春暖,

雨雾朦胧先润街。

年心似箭佳期近,

一岁枯荣在此节。
\end{poem}

去超市一趟所作。这还没到春节呢,雨就开始斜斜地下,周围的建筑物都笼罩在雾中,空气清新湿润,带来春意。盼着过年放假的心情,今年已不剩几天。古时,年过得好坏定调一年的年景,而现在,明年的战略项目Q1迭代也到了关键节点。

\begin{poem}{}{园校}
徐风草荷香,

迟波逸思漾。

叶雨洒苔阶,

抚卷掩幽肠。
\end{poem}

写于母校老校区。

\begin{poem}{}{心屿}
境迁人杳梦未离,

访友抚琴感旧迹。

纷繁汪洋留心屿,

苍茫四顾慰孤旗。
\end{poem}

\begin{poem}{}{心花}
花开招展冬日春,

琉璃如镜映影只。

心思渴望逾墙去,

此身仍在樊笼中。
\end{poem}

此诗为看到朋友圈上一张阳台上的花的照片,评论时随意所作。可按从职场想象外出创业来理解。

\begin{poem}{}{启步}
脂重乏肌显体圆,志虚轻誓总断延。

鞋环衣裤皆齐备,东风乍起旗又偃。

午间会后健房满,晨床迟起无澡时。

轻装放空入夜丛,野径路灯伴胖影。

汗酣血活心宇阔,设标昂首越路人。

面红气短歇不停,耳乐恢宏重拾步。

踝适膝承喘渐平,脚下里程腋成裘。

配速虽缓阶踏实,此战持久莫急求。
\end{poem}

\begin{poem}{}{尝假}
宏愿浅酬临绛海,

千钧垂悬稍息拼。

值此闲景忘紧弦,

夕霞新月淌悦欣。
\end{poem}

\begin{poem}{}{咏茶}
色秋胃暖齿间辽,

壶引杯倾津汇滔。

遍采千株晨滴露,

幽腔远注洗心曹。
\end{poem}

\begin{poem}{}{咏酒}
满壶倾故事,

杯盏碰心声。

此物最忘情,

虽醒犹忡怔。
\end{poem}

\section{丽江香格里拉之旅——2020年09月}

\begin{poem}{}{结游}
行轨攀梯程如网,

即乘云鹤下江川。

水影青赤入眼帘,

旅伴鱼乐出豁然。
\end{poem}

\bigbreak

\begin{poem}{}{穿云}
鹤高白毯铺,

俯首雪山{\xpinyin{冲}{chong4}}。

浅探绒棉绕,

身投薄纱融。
\end{poem}

\bigbreak

\begin{poem}{}{盘山}
葳蕤丛接天,

滂沛褶流泥。

崖下江潺{\textsf 湲},

峰间路迢递。
\end{poem}

\bigbreak

\begin{poem}{}{观湖}
湿雨沁心肺,

和风织锦纹。

岛洲横镜浦,

松影翠氤氲。
\end{poem}

\bigbreak

\begin{poem}{}{湖畔}
山脚连云镜,

汀湾浪息甜。

客游珠玉染,

迁栈远葭蒹。
\end{poem}

\bigbreak

\begin{poem}{}{泛舟}
云穹{\xpinyin{笼}{long3}}九岳,

湖面澈如窗。

风起银鱼跃,

桨落酒窝双。
\end{poem}

\bigbreak

% \newpage

\begin{poem}{}{途遇}
硕石滑坡车戛止,

天晴时早众相嬉。

千钧莫让夜将雨,

悬壁滞停进退危。

队序如安救援顺,

清障重械万心旗。

{\xpinyin{拓}{tuo4}}通险栈道旁护,

出困何能报所贻。
\end{poem}

% \\[2in]

\bigbreak

% \newpage

\begin{poem}{}{踏雪}
目迷云绝顶,

仍上百折梯。

攀陟风箱喘,

冰川回首低。
\end{poem}

\bigbreak

% \newpage

\begin{poem}{}{月谷}
目游碧玺蓝,

心淌雪山泉。

卧石听潺瀑,

意犹汀渚芊。
\end{poem}
\bigbreak

% \newpage

\begin{poem}{}{回忆}
日游仙境景,

夜酌{\xpinyin{少}{shao4}}年事。

方外尽酣欢,

归来锁阁思。
\end{poem}

\end{document}
