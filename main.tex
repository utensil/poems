\documentclass{article}
\usepackage[utf8]{inputenc}
%漢字自動注音
\usepackage{xpinyin}
%漢字注音設置
\xpinyinsetup{ratio = .4,font =\ubufont,multiple={\color{black}}}
%制作诗歌显示环境
\usepackage{tcolorbox}
\usepackage{fontspec}

% \setmainfont{[fzfss.ttf]}
% \setsansfont{[fzfss.ttf]} 
% \setmonofont{[fzfss.ttf]}
% \setCJKmainfont{[fzfss.ttf]}
\setCJKmainfont{FandolFang}
%设置中文正体字体,BoldFont设置粗体和斜体样式对应的字体
\setCJKsansfont{AR PL UKai CN}%设置无衬线样式对应字体
% \setCJKmonofont{[fzfss.ttf]} %设置有衬线样式对应字体

%邊註交互引用
\NewDocumentCommand\pref{m}
{\marginpar{\ding{43}\HGB\ubufontc\hyperlink{#1}#1(\pageref{#1})}}
%正文交互引用
\NewDocumentCommand\lref{m}
{{\HGB\ubufontc\hyperlink{#1}#1(\pageref{#1})}}

\newenvironment{poem}[3] %code={\doublespacing},
{\begin{tcolorbox}[left=1mm,right=1mm,top=3mm,bottom=2mm,enhanced,breakable,pad at break*=.5mm,title=\begin{pinyinscope}\begin{center}\linespread{1.4}\selectfont #2\end{center}\end{pinyinscope},
		colframe=blue!50!black,colback=blue!10!white,colbacktitle=blue!5!myred!10!white,
		fonttitle=\huge\HGBM,coltitle=black,attach boxed title to top center=
		{yshift=-0.25mm-\tcboxedtitleheight/2,yshifttext=2mm-\tcboxedtitleheight/2},
		boxed title style={boxrule=0.5mm,
			frame code={ \path[tcb fill frame] ([xshift=-4mm]frame.west) -- (frame.north west) -- (frame.north east) -- ([xshift=4mm]frame.east) -- (frame.south east) -- (frame.south west) -- cycle; },
			interior code={ \path[tcb fill interior] ([xshift=-2mm]interior.west) -- (interior.north west) -- (interior.north east) -- ([xshift=2mm]interior.east) -- (interior.south east) -- (interior.south west) -- cycle;} }]
      \begin{pinyinscope}
      
      \begin{center}
      
      \LARGE\linespread{1.4}\selectfont #3  %\HGBUM\ZSKS
	}
{\end{center}\end{pinyinscope}\end{tcolorbox}}

\title{冶文斋诗选}
\author{心皿}

\begin{document}
\date{}
\maketitle

\section{丽江香格里拉之旅——2020年09月}

\begin{poem}{}{结游}
行轨攀梯程如网,\\
即乘云鹤下江川。\\
水影青赤入眼帘,\\
旅伴鱼乐出豁然。
\end{poem}

\bigbreak

\begin{poem}{}{穿云}
鹤高白毯铺,\\
俯首雪山冲。\\
浅探绒棉绕,\\
身投薄纱融。
\end{poem}

\bigbreak

\begin{poem}{}{盘山}
葳蕤丛接天,\\
滂沛褶流泥。\\

崖下江潺{\textsf 湲},\\
峰间路迢递。
\end{poem}

\bigbreak

\begin{poem}{}{观湖}
湿雨沁心肺,\\
和风织锦纹。\\
岛洲横镜浦,\\
松影翠氤氲。
\end{poem}

\bigbreak

\begin{poem}{}{湖畔}
山脚连云镜,\\
汀湾浪息甜。\\
客游珠玉染,\\
迁栈远葭蒹。
\end{poem}

\bigbreak

\begin{poem}{}{泛舟}
云穹笼九岳,\\
湖面澈如窗。\\
风起银鱼跃,\\
桨落酒窝双。
\end{poem}

\bigbreak

% \newpage

\begin{poem}{}{途遇}
硕石滑坡车戛止,\\
天晴时早众相嬉。\\

千钧莫让夜将雨,\\
悬壁滞停进退危。\\

队序如安救援顺,\\
清障重械万心旗。\\

拓通险栈道旁护,\\
出困何能报所贻。
\end{poem}

% \\[2in]

\bigbreak

% \newpage

\begin{poem}{}{踏雪}
目迷云绝顶,\\
仍上百折梯。\\
攀陟风箱喘,\\
冰川回首低。
\end{poem}

\bigbreak

% \newpage

\begin{poem}{}{月谷}
目游碧玺蓝,\\
心淌雪山泉。\\
卧石听潺瀑,\\
意犹汀渚芊。
\end{poem}
\bigbreak

% \newpage

\begin{poem}{}{回忆}
日游仙境景,\\
夜酌少年事。\\
方外尽酣欢,\\
归来锁阁思。
\end{poem}

\end{document}
