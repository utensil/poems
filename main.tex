\documentclass{article}
\usepackage[utf8]{inputenc}
\usepackage{csquotes}
\usepackage[unicode,bookmarksopen=true,bookmarksdepth=5,colorlinks=true,linkcolor=blue,pdfusetitle]{hyperref}
\usepackage{bookmark}
% \usepackage{fontspec}
\usepackage{xeCJK}
%漢字自動注音
\usepackage{xpinyin}
% \newfontfamily{\ubufont}[Scale=3]{PingFang SC}
%漢字注音設置
\xpinyinsetup{ratio = .4,multiple={\color{black}}}
%制作诗歌显示环境
\usepackage[skins,breakable]{tcolorbox}
\usepackage{indentfirst}
\setlength{\parindent}{2em}
\setlength{\parskip}{1em}

% \setmainfont{[fzfss.ttf]}
% \setsansfont{[fzfss.ttf]} 
% \setmonofont{[fzfss.ttf]}
% 設置英文字體
% \setsansfont[Mapping=tex-text]{DejaVu Sans}
% \setmonofont[Mapping=tex-text]{DejaVu Sans Mono}
% \setmainfont[Mapping=tex-text]{DejaVu Serif}
% \setCJKmainfont{[fzfss.ttf]}
% \setCJKmainfont{FandolFang}
\setCJKmainfont{Zhuque Fangsong (technical preview)}
% 设置为楷体
% \setCJKmainfont{LXGW WenKai}
% \setCJKmainfont{Dream Han Sans SC}
%设置中文正体字体,BoldFont设置粗体和斜体样式对应的字体
% \setCJKsansfont{AR PL UKai CN}%设置无衬线样式对应字体
% \setCJKsansfont{LXGW WenKai}
% \setCJKsansfont{[fzfss.ttf]}
% \setCJKmonofont{[fzfss.ttf]} %设置有衬线样式对应字体
% \setCJKfallbackfamilyfont{rm}{LXGW WenKai}
% \setCJKfallbackfamilyfont{sf}{KaiTi}
% \setCJKfallbackfamilyfont{tt}{SimHei}

% 洑
% \newcommand{\fu}{氵\hspace*{-0.4em}伏}
% {\xpinyin{\fu}{fu2}} 

%邊註交互引用
% \NewDocumentCommand\pref{m}
% {\marginpar{\ding{43}\HGB\ubufontc\hyperlink{#1}#1(\pageref{#1})}}
%正文交互引用
% \NewDocumentCommand\lref{m}
% {{\HGB\ubufontc\hyperlink{#1}#1(\pageref{#1})}}
%code={\doublespacing}, %\HGBUM\ZSKS
% \newenvironment{poem}[3]{\begin{tcolorbox}[left=1mm,right=1mm,top=3mm,bottom=2mm,enhanced,breakable,pad at break*=.5mm,title=\begin{pinyinscope}\begin{center}\Large\linespread{1.4}\selectfont #2\end{center}\end{pinyinscope},
% 		colframe=blue!50!black,colback=blue!10!white,colbacktitle=blue!5!red!10!white,
% 		coltitle=black,attach boxed title to top center={yshift=-0.25mm-\tcboxedtitleheight/2,yshifttext=2mm-\tcboxedtitleheight/2},
% 		boxed title style={boxrule=0.5mm,
% 			frame code={ \path[tcb fill frame] ([xshift=-4mm]frame.west) -- (frame.north west) -- (frame.north east) -- ([xshift=4mm]frame.east) -- (frame.south east) -- (frame.south west) -- cycle; },
% 			interior code={ \path[tcb fill interior] ([xshift=-2mm]interior.west) -- (interior.north west) -- (interior.north east) -- ([xshift=2mm]interior.east) -- (interior.south east) -- (interior.south west) -- cycle;} }]
%       \begin{pinyinscope}
%       \begin{center}
%       \Large\linespread{1.4}\rmfamily\selectfont #3
% }{\end{center}\end{pinyinscope}\end{tcolorbox}}

\newenvironment{poem}[3]{
% \begin{samepage}
\begin{minipage}{\textwidth}
\begin{pinyinscope}\begin{center}\Large\linespread{1.4}\selectfont #2\end{center}\end{pinyinscope}
% \nopagebreak
\begin{pinyinscope}
	\begin{center}
	\Large\linespread{1.4}\rmfamily\selectfont #3
}{\end{center}
\end{pinyinscope}
% \end{samepage}
\end{minipage}
}

\title{冶文斋诗选}
\author{心皿}

\begin{document}
\date{}
\maketitle

\section{职业生涯}

\begin{poem}{}{心旗}
百挫得来倍珍惜,

阻滞关头意须决。

于无措中求巧解,

为矢寻{\xpinyin{的}{di2}}践其约。
\end{poem}

项目攻坚所作。

\begin{poem}{}{屡战}
筹营后方不甘寂,

杂中炼精视非途。

弃者岂能垮我辈,

再塑箫规{\xpinyin{重}{chong2}}上路。
\end{poem}

团队成员提出离职所作。

\begin{poem}{}{断舍}
远景渐朗继有人,

{\xpinyin{重}{chong2}}燃狼性断舍离。

案头宗{\xpinyin{卷}{juan4}}恍隔世,

再铺新纸染心血。
\end{poem}

组织上安排转岗所作。

\begin{poem}{}{统帅}
倾采思量行果决,

步子错落渐有致。

登高远望酿山河,

绳心养性融诸识。
\end{poem}

有感于团队中长期规划对leader修养的要求而作。

\begin{poem}{}{雨澜}
羡欲摇曳川蚀岩,

志安心谷雨汇洼。

念顽意涩以真灼,

锤炼研磨候缘期。
\end{poem}

羡慕他人际遇所作。

\begin{poem}{}{整装}
卸壳拆戏剖心柳,

意寒行豫沐暖颜。

接骨缝漏{\xpinyin{重}{chong2}}举伞,

铸诚为翼越渊堑。
\end{poem}

重大打击后通过谈心重整旗鼓所作。

\begin{poem}{}{战时}
弛弓引势砺筋骨,

逐猎令出箭夺{\xpinyin{的}{di2}}。

就材起灶拼白刃,

莫待兵成恨狼藉。
\end{poem}

有感于中长期建设赶不上当下应对挑战所需而作。

\begin{poem}{}{泳夜}
一章功成万牍耽,

载愿千钧辙痕浅。

身投静波辉影间,

念出屏囚另开天。
\end{poem}

是夜,与妻女夜泳嬉戏,少顷,妻女还家,独自泳于夜色之中。近日案牍之上,多有事倍功半、愿实背离、罔顾错憾、悬空未决之事,未达使命之感挥之不去,心念中轴转无隙。累时,往往流连美剧,或沉浸手游,不过仍是屏幕的囚徒,以一时之娱回复能量,并未真正休憩。直至回归泳池,目之所及皆是淡蓝辉影,千头万绪,方归于静谧,是以凝成此诗。

\begin{poem}{}{拭剑}
缩岛沉舟谐何妥,

应付终托痂涩喜。

誓绝因庸疑误己,

择伙践志斩荆棘。
\end{poem}

前两句背后的故事是在与人相处、职业发展等方面受挫后迎来转机。后两句是反思过程中的自我怀疑以及作为上的犹豫。

\begin{poem}{}{型势}
宝剑初锋无当者,

细琢日用千面同。

鞘藏温养愈敛涵,

逢境但出破{\xpinyin{长}{chang2}}空。
\end{poem}

在工作上挑战不够的阶段所作。

\begin{poem}{}{伙伴}
成败中道凝风云,

是非成空伙相觑。

执事当惜眼前伴,

共历契心战相许。
\end{poem}

项目调整后有感于要珍惜项目中的伙伴而作。

\begin{poem}{}{投融}
盘综境陌张景怀,

斗转晨曦筹措形。

临渊望岸纵身跃,

漠荒岭峻向北星。
\end{poem}

来到新项目,全力应对全新挑战所作。

\begin{poem}{}{直前}
一叶入秋既别枝,

凭风起舞借扬时。

狡兔恋窟伤末途,

勇往兼济未可知。
\end{poem}

下定决心承担角色出战所作。

\begin{poem}{}{信使}
迟翔青鸟赴淤途,

知候犹诺愿履辛。

纷扰投射居中炼,

蛰坐辗转舒{\xpinyin{长}{chang2}}心。
\end{poem}

承担角色出战后陷入持久战所作。

\begin{poem}{}{失学}
青葱恋文哲,

史{\xpinyin{传}{zhuan4}}鄙尘辉。

当得执绥刻,

方知烛炬微。
\end{poem}

担当重要而复杂的角色,有感于所读历史与传记不够、自身底蕴不足而作。

\begin{poem}{}{将息}
施张百挫知危困,

润物开锋竭所及。

且听雀声荐鸩媒,

收帆稳舵抚蓑笠。
\end{poem}

急流勇退所作。

\begin{poem}{}{闯潭}
起航远望轻风雨,

险鲨恶浪守机舱。

展振蝶翅牵调顺,

云泥扑朔引天光。
\end{poem}

项目遇到未预料的挑战,站好自己这班岗,并有限度地尝试影响全局走向。

\begin{poem}{}{前行}
忠人以诚莫依附,

江湖闯荡不回头。

并肩新遇轻装战,

交游故旧遍神州。
\end{poem}

告别忠爱的领导投入新的人际变局和战场气象而作。

\begin{poem}{}{莫测}
敌彀深陷背来芒,

以身树桃细思{\xpinyin{量}{liang2}}。

浅滩瘠田何相卷,

潮间径筑承钧梁。
\end{poem}

在凶险局势中管理层更替犹豫疑虑后细思定心深耕而作。

\begin{poem}{}{赴蹈}
逐流布子岂轻弃,

舍性收官莫负托。

殷情亭歇盼君归,

挥别善颜履险薄。
\end{poem}

决意不退却回避,出差直面心魔而作。

\begin{poem}{}{难抉}
吞吐万事景无穷,

沉浮此身那堪侵。

一生尽负时光债,

行藏巧{\xpinyin{拙}{zhuo2}}霎时心。
\end{poem}

做关键抉择时,回顾既往,斟酌未来,从辛弃疾《鹧鸪天·不寐》、《重午日戏书》中采字所作。亦作为一次徒步的配诗。

\begin{poem}{}{返初}
林错风穿逡巡雁,

木桥野径过无痕。

碧波石色生机驳,

半{\xpinyin{阙}{que4}}落潮洗心仁。
\end{poem}

徒步见景,有感于一切回到原点而作。\footnote{\hyperref[sec:fan-chu]{赏析}}

\section{压力阴郁}

压力大或心情阴郁时所写,用以倾泻心中负能量,整体偏负面,但起到了心理调节的作用。

\begin{poem}{}{时机}
风云际会无好手,

千里马来非佳期。

熙熙攘攘不得志,

形形色色难如一。
\end{poem}

有感于个人能力/发展诉求与项目/团队需要在时机上的错位而作。

\begin{poem}{}{补天}
每每功亏谁知苦,

夙愿得偿笑颜开。

累年残疾一朝愈,

流失用户{\xpinyin{还}{huan2}}复来。
\end{poem}

修复历史悠久的线上问题所作。

\begin{poem}{}{劳疾}
苦营杂役形枯槁,

重压潜流心焦颓。

鸡血干货有时尽,

执念愿景转成灰。
\end{poem}

工作压力过大所作。

\begin{poem}{}{多艰}
白日连轴转,

夜晚噩梦缠。

拿起庶几成,

放下乱成团。
\end{poem}

多艰之意分为事多、事艰。前两句说的是事情多,夜晚噩梦往往不是那些恐怖片桥段而是工作场景, 待办事项白日做不完以致入梦。

白天比晚上好过,因为白天有无数的事情涌来,你也会要立即响应,没有空给多余的情绪;然而晚上,所有的未解问题和今天新产生的问题,会再进行各种化学反应,产生深层次的情绪效应,而所有的措施都不能采取,只能在失眠或梦中徒劳,醒来又是头痛欲裂、几无寸进。

后两句说的是事情艰难,而且套用拿得起放得下的俗话,说拿起来的没几件成功,放下的事情不理会一段时间就会乱成一团,反过来也说明事情即使以时间管理的技巧分优先级做也难以应对。

当仅仅既重要又紧急象限的事情就已经耗完或者超出团队的所有时间,团队不可能有时间做到所谓的时间管理或者做事模式优化,只会深陷恶性循环无法自拔。团队过饱和之后,激励不足的成员就会离去,激励足够或者内驱的成员就会累死。

\begin{poem}{}{{\xpinyin{还}{huan2}}家}
连轴苦战为年安,

纷至迭来案无垠。

身在樊笼心企渴,

一朝放飞疚难平。
\end{poem}

过年前忙完把工作交接给值班同事后回家过年路上所作。但其实因情况发展始料未及,年初三又赶回公司。

\begin{poem}{}{肩钧}
扶犁耕新野,

授将镇旧疆。

奔走拨千斤,

失察辛莫赎。
\end{poem}

前两句的背景是需要兼顾新老团队和新老项目,老项目可更多授权,新项目则需要事必躬亲。后两句有感于自己在精力有限的情况下兼顾两端,担心出现疏漏造成不可弥补的问题。

\begin{poem}{}{未竟}
殚精竭虑周不全,

连迭赶忙未如期。

年年规划竟半功,

岁岁值守起新波。
\end{poem}

有感于工作每个方面都只做了个半好而作。

\begin{poem}{}{低效}
勉力营馨难兑诺,

案牍烦忧蚀图景。

熙攘无处解孤渴,

思华星火成烟烬。
\end{poem}

由于未能很好地兼顾家庭与工作,更深地感受到个人的孤独而作。

\begin{poem}{}{衣悟}
德责勉尽咎愧仍,

两全难能何缚己。

沧粟漂萍负块垒,

不如一{\xpinyin{骑}{ji4}}驰心野。
\end{poem}

由于未能很好地兼顾家庭与工作,寻求洒脱的心境而作。

\begin{poem}{}{岩浆}
时偿积欠缺填勉,

凝途失涩兴阑珊。

奔流方遒黯如{\xpinyin{没}{mo4}},

趣中汲静思里安。
\end{poem}

寻找支撑忙碌工作的动力而作。

\begin{poem}{}{望空}
立身荒谬耕心意,

倾注殷切抹能轻。

因果涟漪噬回{\xpinyin{还}{huan2}},

于墟棘中缮草亭。
\end{poem}

绸缪已久的规划被突发安排推翻,心中失落而作。

\begin{poem}{}{掌中}
势成相左湮持庸,

幽阵摊推守驰驱。

颠簸诡谲栖疏惬,

井井施条焕遗墟。
\end{poem}

在人际纷争中寻求内心平静而作。

\begin{poem}{}{齿轮}
念执临穴{\textsf 洑}水行,

触域及高知位险。

十方错切拢散盘,

千辛半阙耕寥田。
\end{poem}

在凶险局势中寻求内心平静而作。

\begin{poem}{}{井观}
遐瞻垂问计何定,

款{\xpinyin{曲}{qu3}}裁编众竭泽。

曙影乍开天幕收,

下稍究尽命无择。
\end{poem}

有感于某业务前景难以挽回而作。此作用典较多,稍晦涩。

\begin{poem}{}{沉沦}
食色惰慢七罪四,

贪嗔疑虑五毒三。

身沾万点尘世红,

心缠半丝天际蓝。
\end{poem}

有感于自己在人性缺陷中沉沦而作。

\begin{poem}{}{席散}
萃聚一堂开新行,

求索雕琢竟曲终。

来途荏苒同舟济,

前路依稀各乘风。
\end{poem}

有感于业务调整尘埃落定、曾在一起的团队各奔前程而作。

\section{情感与家庭}

\begin{poem}{}{心印}
盈盈倩影跃,

嫣嫣美目盼。

各归意伶俜,

寤寐思身畔。
\end{poem}

有所爱恋,有所思念而作。

\begin{poem}{}{夜会}
别前相约衷情诉,

厮磨飞霞染耳梢。

托付此心{\xpinyin{长}{chang2}}久远,

怎堪行去念思滔。
\end{poem}

别前相见后,启程途中回想所作。\footnote{\hyperref[sec:ye-hui]{赏析}}

\begin{poem}{}{心意}
唇间春意露,

身后臂膀环。

安居甜蜜意,

携手浪迹心。
\end{poem}

世间男女偶遇相恋后,虽情意绵绵,心中所憧憬的,却是不一样的幸福。小龙女要的是终老古墓,杨过要的是携手浪迹天涯。女性常以家庭为终极价值、归宿,男性以家庭为事业生涯的基石、港湾。这种爱恋的引力和心意的相背,构成了长期相处的张力。于花市看见一对甜蜜情侣,心有感而作。

\begin{poem}{}{喜临}
同事儿女接连诞,

意犹忐忑羡天伦。

忽如一夜豆发芽,

山花烂漫面春风。
\end{poem}

妻子有喜所作。

\begin{poem}{}{十月}
纸{\xpinyin{迹}{ji4}}浅浅映愿深,孕{\xpinyin{吐}{tu4}}翻腾喜含酸。

唐筛排查换心宽,四维彩超初见形。

萌童海报环四壁,典{\xpinyin{乐}{yue4}}诗谣开鸿蒙。

腹隆曲美留存照,五谷杂粮控{\xpinyin{血}{xue3}}糖。

腰背腿疼卧难安,胎动如悸渐踢蹬。

脐绕不再位已正,入盆待命盼宫缩。
\end{poem}

妻子十月怀孕所作。

\begin{poem}{}{疹热}
反侧眠未安,
煎熬口难言。

楚泪湿枕席,
内焚红耳弯。

父母千{\xpinyin{重}{chong2}}忧,
诊者一言宽。

轻身脱病沼,
复享天伦欢。
\end{poem}

女儿发热所作。

\section{哲思禅意}

\begin{poem}{}{孤僧}
花市灯华人如潮,

我缀其中如孤僧。

皮色红尘愿尽历,

只求得透空中道。
\end{poem}

春节花市所作。

前两句是闹中孤寂,华景空心。

后两句中,色相狭义指可被感知的世界,广义指围绕主体的符号秩序,是心相不是实相。皮相则是借指实相。存在是不可能的,对存在的意识更是不可能的,然而我们却确实活在双重不可能中。不可能的存在唯一诱人的是它的之所以可能,亦即道。没有道,一切便都是空,都是色。

\begin{poem}{}{花泥}
世间本来无有我,

何必以我求忘我。

悲喜触识{\xpinyin{着}{zhuo2}}此在,

百年之后化花泥。
\end{poem}

见“求忘我”语所作。

在贪著妙触的基础上,又增加了“识”,代表着求知欲,代表着知识在“我”这里的融合。花泥是有序生命体的消散,也是新生生命体的养分。

所以,虽然“我”是空的,却融入了更大的生生不息的空。这样的人生图景,美丽而意足。

\begin{poem}{}{自然}
何智逐尘功,

本真望莫渴。

求索生新梦,

实{\xpinyin{相}{xiang4}}在心{\xpinyin{豁}{huo4}}。
\end{poem}

采撷意译创作自叶芝的诗:

\begin{displayquote}
Then nowise worship dusty deeds,

Nor seek, for this is also sooth,

To hunger fiercely after truth.

Lest all thy toiling only breeds

New dreams, new dreams; there is no truth

Saving in thine own heart.
\end{displayquote}

\begin{poem}{}{觉来}
吾侪心事{\xpinyin{长}{chang2}}流水,

停觞自醉梦渊明。

富贵无味不应堪,

东山饶起为苍生。
\end{poem}

以出世之心境入世,从辛弃疾《水龙吟·老来曾识渊明》中采字所作。

\section{心情与事件}

\begin{poem}{}{相寻}
走马前边行,

流连人{\xpinyin{没}{mo4}}漫。

众里茫相寻,

一枝蝴蝶颤。
\end{poem}

花市人多走散寻回所作。取“众里寻他千百度,蓦然回首,那人却在,灯火阑珊处”的意境。

\begin{poem}{}{年心}
冬寒指日{\xpinyin{还}{huan2}}春暖,

雨雾朦胧先润街。

年心似箭佳期近,

一岁枯荣在此节。
\end{poem}

去超市一趟所作。这还没到春节呢,雨就开始斜斜地下,周围的建筑物都笼罩在雾中,空气清新湿润,带来春意。盼着过年放假的心情,今年已不剩几天。古时,年过得好坏定调一年的年景,而现在,明年的战略项目Q1迭代也到了关键节点。

\begin{poem}{}{园校}
徐风草荷香,

迟波逸思漾。

叶雨洒苔阶,

抚卷掩幽肠。
\end{poem}

流连母校老校区所作。

\begin{poem}{}{心屿}
境迁人杳梦未离,

访友抚琴感旧{\xpinyin{迹}{ji4}}。

纷繁汪洋留心屿,

苍茫四顾慰孤旗。
\end{poem}

访友忆昔所作。

\begin{poem}{}{心花}
花开招展冬日春,

琉璃如镜映影只。

心思渴望逾墙去,

此身仍在樊笼中。
\end{poem}

看到朋友圈上一张阳台上的花的照片,评论时随意所作。可按在职场里想象外出创业来理解。

\begin{poem}{}{启步}
脂重乏肌显体圆,志虚轻誓总断延。

鞋环衣裤皆齐备,东风乍起旗又偃。

午间会后健房满,晨床迟起无澡时。

轻装放空入夜丛,野径路灯伴胖影。

汗酣{\xpinyin{血}{xue3}}活心宇阔,设标昂首越路人。

面红气短歇不停,耳乐恢宏{\xpinyin{重}{chong2}}拾步。

踝适膝承喘渐平,脚下里程腋成裘。

配速虽缓阶踏实,此战持久莫急求。
\end{poem}

夜跑减重所作。

\begin{poem}{}{尝假}
宏愿浅酬临绛海,

千钧垂悬稍息拼。

值此闲景忘紧弦,

夕霞新月淌悦欣。
\end{poem}

十一长假,中间捞少数几天,短途旅游,浅浅品尝这长假。

自然回想起《滕王阁序》、《赤壁赋》、《兰亭集序》中的场景:“仰观……俯察……游目骋怀……耳得之而为声,目遇之而成色……穷睇(dì)眄(miǎn)于中天,极娱游于暇日。天高地迥,觉宇宙之无穷;兴尽悲来,识盈虚之有数”。

\begin{poem}{}{咏茶}
色秋胃暖齿间辽,

壶引杯倾津汇滔。

遍采千株晨滴露,

幽腔远注洗心曹。
\end{poem}

心中积郁,饮茶得解而作。

\begin{poem}{}{咏酒}
满壶倾故事,

杯盏碰心声。

此物最忘情,

虽醒犹忡怔。
\end{poem}

醒酒所作。

\begin{poem}{}{鱼缸}
敦身鳍缓箭穿梭,

高游低栖尽资餐。

但有海神立星国,

扬鳞同销万古{\xpinyin{难}{nan4}}。
\end{poem}

吃海鲜前于鱼缸中见大鱼生猛姿态,有感于族类不被圈养役使有赖于强者而作。

\begin{poem}{}{羽决}
碳框{\xpinyin{弹}{tan2}}杆趁手柄,

虎口以力拇指巧。

移步顺腰肩背耸,

臂挥腕送掌心饱。

开肢侧身面球轨,

九宫八向切拍{\xpinyin{挑}{tiao3}}。

太极圆柔咏春袭,

五禽起伏金箍绞。
\end{poem}

使羽毛球拍如有剑诀所作。

\begin{poem}{}{得遇}
岁月侵蚀人微渺,

褴褛凡衫挂挺枝。

何逢契魂由衷励,

残损抚遍心伤弥。
\end{poem}

有感于蹉跎岁月中得遇知己,采撷意译创作自叶芝的诗:

\begin{displayquote}
An aged man is but a paltry thing,

A tattered coat upon a stick, unless

Soul clap its hands and sing, and louder sing

For every tatter in its mortal dress.
\end{displayquote}

\begin{poem}{}{出戏}
层峦碧嶂雪狮欢,

暖浪濯足惊溅寒。

潮去漩{\textsf 洑}陷泥滩,

风来意阔尘嚣安。
\end{poem}

海边休假戏水,暂时出脱释怀所作。

\begin{poem}{}{滋长}
迥廊椰玉柱,

藤叶翠扶疏。

柔蔓垂萦纡,

谁知雨前初?
\end{poem}

雨中漫步绿廊,有感于人际故事缘起而作。

“谁知雨前初”,意为如今盘根错节的藤蔓滋长于一场场雨中,后来人又有谁知道在这一切发生之前的初态?

\begin{poem}{}{遴选}
较艺抡材衡鉴内,

丹忱筋韧网罗中。

欣逢翘楚登时纳,

排布分忧共僦功。
\end{poem}

在“旰宵汲汲”与“夙夜孜孜”中给专场招聘会的寄语,改写自宋·赵恒《又将放榜》,用典较多。

\begin{poem}{}{纵擒}
犁滩浪袭贝踪现,

蚌蠕壳扬钻深忙。

眼疾步挪手啄沙,

掌心翻覆返海乡。
\end{poem}

海边拾贝所作。

\begin{poem}{}{哧溜}
恢恢扑蝶网,

悠悠淌冰溪。

浅底现形{\xpinyin{迹}{ji4}},

倏而{\xpinyin{没}{mo4}}湍泥。
\end{poem}

溯溪捉鱼所作。

\section{云南丽江香格里拉之旅}

这趟旅行,文思泉涌,系列诗作,单成一辑。

\begin{poem}{}{结游}
行轨攀梯程如网,

即乘云鹤下江川。

水影青赤入眼帘,

旅伴鱼乐出豁然。
\end{poem}

出游前想象所作。

\bigbreak

\begin{poem}{}{穿云}
鹤高白毯铺,

俯首雪山{\xpinyin{冲}{chong4}}。

浅探绒棉绕,

身投薄纱融。
\end{poem}

乘机降落时观云所作。

\bigbreak

\begin{poem}{}{盘山}
葳蕤丛接天,

滂沛褶流泥。

崖下江潺{\textsf 湲},

峰间路迢递。
\end{poem}

盘山公路上所作。

第一第二句写车上看到的景色:第一句通过写山上茂密的树丛直接与天相链接,来体现海拔之高,离天之近;第二句写这里一下雨就经常泥石流,青山上土黄色的伤痕触目惊心。

第三第四句都是写坐大巴开盘山公路的感受:第三句写因为路窄,车非常靠近悬崖,江水仿佛直接在车下流淌,江水流得缓慢,而车速很快;第四句写山势险峻,盘山公路十分曲折,在车上感觉到来回扭拗的离心力。

\bigbreak

\begin{poem}{}{观湖}
湿雨沁心肺,

和风织锦纹。

岛洲横镜浦,

松影翠氤氲。
\end{poem}

于观景台观泸沽湖所作。

\bigbreak

\begin{poem}{}{湖畔}
山脚连云镜,

汀湾浪息甜。

客游珠玉染,

迁栈远葭蒹。
\end{poem}

在泸沽湖畔浅滩,有感于湖边建筑后迁以保护环境而作。

\bigbreak

\begin{poem}{}{泛舟}
云穹{\xpinyin{笼}{long3}}九岳,

湖面澈如窗。

风起银鱼跃,

桨落酒窝双。
\end{poem}

泛舟泸沽湖所作。

\bigbreak

% \newpage

\begin{poem}{}{途遇}
硕石滑坡车戛止,

天晴时早众相嬉。

千钧莫让夜将雨,

悬壁滞停进退危。

队序如安救援顺,

清障重械万心旗。

{\xpinyin{拓}{tuo4}}通险栈道旁护,

出困何能报所贻。
\end{poem}

泸沽湖返丽江途中遇山体滑坡被困得出所作。

% \\[2in]

\bigbreak

% \newpage

\begin{poem}{}{踏雪}
目迷云绝顶,

仍上百折梯。

攀陟风箱喘,

冰川回首低。
\end{poem}

带着高原反应,登顶玉龙雪山所作。

\bigbreak

% \newpage

\begin{poem}{}{月谷}
目游碧玺蓝,

心淌雪山泉。

卧石听潺瀑,

意犹汀渚芊。
\end{poem}

游蓝月谷所作。

\bigbreak

% \newpage

\begin{poem}{}{回忆}
日游仙境景,

夜酌{\xpinyin{少}{shao4}}年事。

方外尽酣欢,

归来锁阁思。
\end{poem}

出游结束后,回归日常工作仍未收心所作。

% \section{诗词摘抄}

% \begin{poem}{}{京路夜发}
% 晓星正寥落,

% 晨光复泱漭。

% 犹沾余露团,

% 稍见{\xpinyin{朝}{zhao1}}霞上。

% 敕躬每{\textsf 跼}{\textsf 蹐},

% 瞻恩唯震荡。

% 行矣倦路{\xpinyin{长}{chang2}},

% 无由税归鞅。
% \end{poem}

% \begin{poem}{何 逊}{穷乌赋}
% 嗟穷乌之小鸟,意局促而驯扰。

% 声遇物而知哀,翮排空而不矫。

% 望绝侣于霞夕,听翔群于月晓。

% 既灭志于云霄,遂甘心于园沼。

% 时复抢榆决至,触案穷归,\phantom{穷归}

% 若{\xpinyin{中}{zhong4}}气而自堕,似惊弦之不飞。

% 同雉埘而共宿 ,啄雁稗以自肥 。

% 异海鸥之去就,无青鸟之是非 。

% 岂能瑞周德而丹羽 ,感{\xpinyin{燕}{yan1}}悲而素晖。

% 虽有知于理会 ,终失悟于心机。
% \end{poem}

% \newpage

\section{代后记:在日常里写旧体诗的一点体会}

受到语文教育古诗词熏陶的每个人,或许都尝试过用旧体诗来书写,至少也尝试过改编原诗,或者写些打油诗,这个阶段没有太高门槛。但若要继续往前,文化积淀、诗词格律形成了技术门槛,鉴赏水平、敏锐的心灵、意象、诗意、思想等,形成了软性门槛。所以,在日常中,诗更多是用来吟诵、赏析、引用,如果引用的时候贴合情境,那就已经是很不错的运用了。也有爱好者形成的写诗的群体,但对于彼此写诗的水平,看法可能很不一致,局面可能是,作者孤芳自赏,圈中人不以为然,而我们只能在圈外不明觉厉。

旧体诗,在日常里,只能处于这般只读的状态,无法拥有交互和书写的体验,是极其可惜的。是否可以从一个比较低的起点开始,尝试融化部分门槛,让旧体诗成为自己可以在日常中使用的一种文体,表达自己真情实感呢?我发现是可以的。

可是,从根子上说,那么多文体可供选择,为什么要写旧体诗呢?

在我还没有习惯写旧体诗的时候,我曾写道:``我觉得诗意和诗的形式是存在的。但我觉得诗和诗人是不存在的。诗不能独立作为写作对象而存在,它,正如词是诗余一样,是文余。诗意也是意义的余,一种副产品。它是意外而美丽的凝结,它的纤巧无法刻意。它是那个在做着想着其它事情的人,无意中给后人留下的词句。所以我觉得,很多单独成诗的,反而不是诗。洒洒行文之间,突然读到的那一两句惊心的句子,却常常是更美的诗。它们正如嵌在山岩里的玉璞,是诗的原初形态。''所以那时我在日常里,更喜欢的是在行文中自然带上诗意,而不是专门以诗作为一种文体。

后来,在后微博和朋友圈的时代,我会特别喜欢用旧体诗来作为一种表达方式。整体来看,它在字数规则上比较简单固定,不像长短不一的词,而无论是内容上和形式上,细节处都有很多打磨的空间。为了在短小的篇幅中表达丰富的意思,它有一种追求言简意赅、字字珠玑的精致,是非常浓缩的精华,事后读来,有点就像《三体》里质子大小的智子的低维展开一样,意蕴无垠、回味无穷。

因为我的工作很忙,并没有时间像学生时代那样大段地去记录发生的故事和所思所想,我写诗更多地是作为一种记录生活与调节心态的方式:那些一时无法忘怀的事情与感受,如果我把它凝结到一首诗中了,就仿佛可以由它来替我铭记和承载那些感受,而我可以回到日常中,继续扮演我的角色,只在后来某一天,在可以有时间岛屿的奢侈时,回到诗前,重新展开那段经历和那片心情。

诗作的水平不要紧。开始写时,总不能去和那些经过历史大浪淘沙留下来的经典去比较。《香菱学诗》的故事里讲的写诗的道理,都是很好的,但香菱的诗到第三首便有那般水平,是不科学的,甚至要达到第一首那样成功地堆砌前人词藻,也是殊为不易的过程。写诗,和写任何文字一样,更多地是一种只与作者自身相关的体验,将作者的见闻、感受、思想,凝结到诗中,自己在时过境迁之后读来,还能唤起当时的感受,那便已经是成功。

人类的悲欢,不说不相通,至少在大多数情况下,个人的体验都是非常独有,而难以唤起其他人的共情的,更别提像经典一样,跨越时空感染后人。这便如唱歌,无论基础如何,自己K歌,总是可以唱好、尽兴的,但若要在旁者听来好听,甚至成为歌星,那完全是另外一回事。舞蹈也是每个人都可以使用的一种舒展变换自己肢体的艺术,只要自己跳着舒服就好,但若要在旁人眼里形成美感,则需要严格的基础功训练。

所以,要开始,就要先以习作的心态来写诗,只要能够有效地凝结了那一刻的时空与自己,在词句、立意任一维度稍有可取之处,便很好。

当然,如果自己那一刻根本没有和诗相关的灵感,或者要表达的内容,根本不适合诗,或者不适合旧体诗,那就要果断选择其他的文体,别来招惹诗。也不要强行穿越到古人的时空里,以古人、别人的主题写诗。可以尝试命题写诗,但命题必须从自己确实存在的体验中来找,近的可以是旅游时见到的景观,远的也可以是刚看了一本书或者一部剧,有所感的故事。

旧体诗作为一种文体,要求作者拥有敏锐的观察和分析周遭的能力,对内心的思想和情感也要非常敏感,然后需要具备非常感性和具象的想象力,还需要能够有一定的强迫症或者秩序感,以及盈余的智力,来去探索诗的格律要求的音韵美和形式美。这些特质是天生的,但我相信每个人都或多或少具备。

某种意义上,格律是一个普通人写旧体诗最大的心理障碍。首先要学会自身与之相处的方式,才能真正动笔去写旧体诗。作为一个不喜欢繁复的记忆负担,也不喜欢为规则所约束的人,我在相当长的时间里,都无法走近格律。但我隐约知道,它是要去趋近的正道。我曾经借阅或者买过许多讲诗词格律的书,都很好,很严谨,很有文化,我却始终不得其法。说不清是不是只是第七个饼,最终在遇见《格律诗新探(唐人引以为豪的当时体)》》之后,突然有了感觉。也是在那之后,我才突然觉得写旧体诗的世界对我不再关闭,而成为了可以交互的对象,也开始留下了今天看来还算诗作的尝试。

我找到的感觉是,既然是习作的心态,那么在格律方面,其实可以有很大弹性。``不要以律害意''首先可以确立是正确的最高原则,虽然它经常被用作不去好好学习格律的借口。很多时候,个人功力没到,想表达的意境,就是没法找到合乎音韵、平仄的字词来去表达,如果的确是这种情况,这首诗,只能炼到这儿了,要承认,把好的、已经完成的部分,保留下来。当然,格律是一定要学习掌握的,写诗时也应该瞄准最严谨的近体诗的标准,在用韵方面,首先瞄准平水韵,然后再根据力所能及,酌情降级到普通话的韵,甚至有时候意象、词句觉得好,放弃押韵也是可以的。最后这种情况,可能很有争议,但如果去翻阅一些成名诗人的全集,就会偶尔发现这样的例子,所以不必执着。

弹性并不意味着降低自我要求。诗意犹在的时候,可以动用盈余的智力,去尝试贴合格律。格律本身是对音韵美、节奏美的客观规律的总结提炼,是一个应该趋近的标准。本质上,是因为前人已经有了很多诗作,在宽松的要求下已经有了很多佳作,所以,近体诗意义下的格律才诞生出来,作为专业人士进一步追求卓越的指引。它是一个关于形式美的文字游戏,是一个求解方程的过程,它蕴含的是一种解谜的乐趣。如果在立意、意象等方面已经恰好有所得,如果可能,为何不进一步打磨其形式,说不定就可以再美一点?当然,艺术的追求是一个不确定是否收敛的过程,如果力竭,那么就要安于所得。诗,作为上天的赠与,它的美丽总是有限的。在这首诗里未达到的,还可以有下一首的机会。

我非常早期的一首诗作,就是连押韵都没顾及:

\begin{quote}
唇间春意露,身后臂膀环。

安居甜蜜意,携手浪迹心。
\end{quote}

创作背景是:

\begin{quote}
世间男女偶遇相恋后,虽情意绵绵,心中所憧憬的,却是不一样的幸福。小龙女要的是终老古墓,杨过要的是携手浪迹天涯。女性常以家庭为终极价值、归宿,男性以家庭为事业生涯的基石、港湾。这种爱恋的引力和心意的相背,是构成了长期相处的张力。于花市看见一对甜蜜情侣,心有感而作。
\end{quote}

虽然音韵平仄皆是完全离谱,此作有其他方面的可取之处。形式上,前两句是4字名词+1字动词的结构,后两句是4字形容词+1字名词的结构,分别都算对仗整齐。前两句写具体看到的情形,后两句升华写感受,一三写女性,二四写男性,结构上对应。个人理解,这是比较典型的,早期尝试可以达到的形式美状态。用词上,不会过于日常口语化,也并没有怎么文雅,其语感大概在放在诗里不至于不伦不类,或者有点像成语的程度。

这首诗,有画面感,有情感,有思考,有一定的形式结构之美。对于初写诗者,已经照顾到了这么多方面,夫复何求?

尤其是在画面感与思想之间的平衡上,是不容易的。有的时候,全诗都写成了思想,有的时候,全诗都写成了画面。

前者的早期例子是(创作背景是工作上有很大困难要去克服,才能实现承诺了要做到的上级的要求):

\begin{quote}
百挫得来倍珍惜, 阻滞关头意须决。

于无措中求巧解, 为矢寻的践其约。
\end{quote}

全诗就用了一个箭的形象,其他全是思想。

后者的早期例子是(创作背景是回到母校校园):

\begin{quote}
徐风草荷香, 迟波逸思漾。

叶雨洒苔阶, 抚卷掩幽肠。
\end{quote}

全诗都是写景,最后有一个幽肠的情感,但与景色并无关联,只是的确是我在校园里想起了很多过去的事情产生的情感。

除了形象和抽象之间的平衡之外,也需要注意``乐而不淫,哀而不伤''。我有的诗写于工作压力很大的时候,整体风格就极度阴暗抑郁,比如:

\begin{quote}
苦营杂役形枯槁, 重压潜流心焦颓。

鸡血干货有时尽, 执念愿景转成灰。
\end{quote}

这几个例子,我们不去讨论它们是否是佳作(事实上,我的诗作里勉强能竞争佳作的都偏后期,也都更合格律,或许未来才能登大雅之堂,但前期的时空、感受、稚嫩而认真的尝试,都一去不复返了,所以在本文的语境中,反而更有参考意义),但它们都起到了那一刻我希望它们起到的作用:写出来,忘记它,未来回味。

从这几个例子的用词也可以看出,我不是很追求白居易那``老妪能解''式的明白易懂。我更追求把意思用字词准确地捕捉和刻画出来,只有掌握了那一刻体验真相的我自己能判断用词的准确性,或者像《红楼梦》那样用曲笔藏起来,我只蜷缩在层层外壳之内。诗作读起来,需要一个字一个字地理解,辨识其中的背后深意,可以是需要另外解释才能够体会。我相信写诗是一种编码的过程,而解码是有门槛的。

我偶尔会在诗中用上生僻字或者比较生僻的连绵词。这些字或者词,看到时我们隐约知道是什么意思(因为在别的语境下似乎见过),写的时候你会模糊地想起想用这个词,不一定回忆得起它的完整发音,也不一定记得它的完整字形,所以直接用不出来,但凭着那种感觉经过一定的搜索,能够找到它。当它合适这首诗时,我想起了它,找到了它,再把这块宝石镶嵌在了我的诗中,从此它便是我的,而不再是故纸堆里已死的存在和今人记忆的负担。

可是,如果如此晦涩,岂不是完全成为了无法阅读的孤独的密文?我相信诗是可以有很多层的,虽然深入解读下去,诗的细节能藏住我的很多想法,但乍读起来能够像看一幅画的缩略图一样,领略到画整体的意境和情感。我心目中追求的诗作的极致是:看起来是一个样,稍微解读一下,就到达了对表象的否定,但再深入到内核,却又与表象相协调一致,形成否定之否定,或者诗吃起来一开始是一种味道,但每舔开一层,都有新的五味。

这便是一种意义的编织,它不是一种刻意,而是因为诗人在那一刻经历的认知与情感,本身就是没有一致性、没有逻辑性的意识流,是一种丰富多层次的体验,所以,也可以把它注入到诗的多层、多向的编织结构中。正如我在《雕琢》里面写道:``雕琢最是不经意间的投入,不若刻舟求剑的执着。世间的事情,若作为机械的任务,难得发指,若意起雕琢起来,完成时只觉得自然而毫无阻滞,亦察觉不出落笔之繁复。''

这个也反向对诗的句式提出了要求。我不喜欢过于口语化,像一句话一样的直抒胸臆的句式。这样的句式过于浪费字数,一句诗就只能表达一种意思。这个时候,电影里蒙太奇的手法是非常好的。我们不需要点出所有的主语、宾语,也不需要直接地把事物之间的细节联系用动词表达出来。我们只需要把每件事物或者事情用直接或隐喻的方式刻画或代表,然后把它们摆在一起,布置成一片天地,让人类那渴求建模的大脑来完成诠释。如果我们真的要去使用动词,也应该是针对这个画面整体的感受,主语和宾语都不再是这幅画中的某个组成部分,而是整张画。这个手法背后,也是诗对我们的想象力的要求,某种意义上,想象力正是对于潜在的可能联系的发现。

不重字的规则,我一直都特别在意,每一首都坚持。全诗中用字不能重复,诗中原则上不能出现标题中有的字。对于要描写的事物和情感,我们不要直接呼唤它的名字。诗中处处出现它的影子,但就是不要直接出现它的本体。意思可以重复,形成结构性的呼应,但其形态必须变幻。新诗、长诗、歌词或者《诗经》,更适合咏叹式的冗余重复,但旧体诗在绝大多数情况下不适合。唯一允许的例外,是使用叠字的连绵词,或者顶针结构,但适合的场景极为罕见,也可能会冲淡诗的浓度。

“诗言志。”“文章憎命达。”“诗歌是生命的自然流淌。”工作中的我们,已经远离了少年时代的理想、情结、或者拥有其他名字的初心,但也从未忘却,偶尔燃起愿景、豪情或是洒脱。(“倾采思量行果决,步子错落渐有致。”“临渊望岸纵身跃,漠荒岭峻向北星。”“沧粟漂萍负块垒,不如一骑驰心野。”)在压力巨大的工作中,我们会遇到许多挫折。(“殚精竭虑周不全,连迭赶忙未如期。”“一章功成万牍耽,载愿千钧辙痕浅。”)在职业生涯的发展中,我们会遇到困顿与曲折,也会面临需要勇气的抉择。(“鞘藏温养愈敛涵,逢境但出破长空。”“狡兔恋窟伤末途,勇往兼济未可知。”)在人际关系中,我们有伤心和温暖的时刻。(“誓绝因庸疑误己,择伙践志斩荆棘。”“执事当惜眼前伴,共历契心战相许。”)在情感或家庭生活中,我们有天伦之乐,有各种突发变故,也有许多心事。(“同事儿女接连诞,意犹忐忑羡天伦。忽如一夜豆发芽,山花烂漫面春风。”“熙攘无处解孤渴,思华星火成烟烬。”)在聚餐、旅游、K歌等娱乐活动中,我们拥有过刹那的放开的自我。(“满壶倾故事,杯盏碰心声。此物最忘情,虽醒犹忡怔。”)这一切,都可以酝酿成诗,这些体味,都可以入诗。在相当长的时间里,我都是相隔数月,才凝成一首诗,标志性地记录一段生涯。或者偶尔,因为在关键节点上,我的内在挣扎或者反思很多,一件事产生多个截然不同的想法,会连续有几首不同的诗。又或许,一次旅游,就促使诗兴大发。每到一处景观,产生感受,用好赶路的独处时间,便是一首诗。甚至一处景观,因为不同角度产生的感受迥异,便写出多首诗。一首诗的四句,可以正好一一对应一起发在朋友圈的四张图。有时可能从某句聊天触发诗兴,到一挥而就,再修改到合乎格律,只用了十几分钟。每首诗,我要写的,既是那一刻,是那一个具体的事情或者事物,又会不由自主地折射出自己身处的整个局,或是结晶了跨越半年或者数年这样的时间尺度的零碎心绪。

这便是诗的选题。旧体诗,的确是一种很有意思、很特殊的载体,它的确也能承载一个现代人,一个从事互联网技术工作的人的生活与思想。当然,它只是无数的载体之一。公域的微博或者私域的朋友圈,可以有很多丰富的形式:在媒体上,可以有照片、视频、音乐;在文本上,可以有短句,可以有散文,可以有新诗,可以有旧体诗,可以有英语或者法语,还可以有UnicodeArt或者Emoji。在这个光怪陆离的世界中,旧体诗可以有它的一个位置,一个像茶一样的位置,成为我们日常表达的一部分,观照自我和世界。

在现代写旧体诗,自然有一些新的方法,实际上也不可避免会用到一定的科技。科技的引入,并不会削弱诗本身的古典意味。毕竟科技只是工具,人才是掌握工具的主体。就算人类进入了星际大航海或者银河帝国的时代,那些情怀依然与古人没有什么不同,一样是一个有着自身有限性,但在思想和情感上却可以无垠的有情生命体,面对这个充满未知和不确定性的世界,所发出的面向永恒时间的短短心声。

在这个诗艺没落的时代,一个普通人,通常不会对古籍耳熟能详,不会对韵典烂熟于心,也不会在肌肉记忆中刻印那么多格律的规则。一个手机App能起到很大的帮助,消融这里的门槛。以前我用过``诗词吾爱'',后来主要是用``搜韵''。

诗兴来的时候,我会用纸笔先涂涂画画,纸上会散落半句、字词、意象等混沌的脑图,渐渐可以用行书写为一首完整结构的诗,又涂涂改改,在自有语感的指引下,寻找更贴合和表达和字词。到这里,完成了自身能力支撑的前半截创作。然后把诗打字到手机里,复制到搜韵的``创作''功能里,让它校验用韵、平仄,并且给出修改建议。

``搜韵''校验和建议的功能并不是很完善,但可以适应和善用:

1、它能识别出当前的用韵,但不能直接点击跳到对应的韵部,韵部也无法搜索,需要单独去肉眼找韵部;

2、对于多音字的平仄,它给出分别的发音和词义,让自校,但没有明确提示应该平还是应该仄,需要手工输入一个明确的平仄字来确认;

3、对于出韵、或者不合平仄的字,根据前后文,它能提示一些替换的组合,但通常因为意义方面的原因不能使用,不过,作为一个联想的种子是很好的;

4、韵典中字的排序,没有逻辑性,没有按照古诗词中出现的频率来,导致常用字和生僻字混杂在一起,找合适的字的体验不够流畅,有点打断心流;

5、缺乏流畅地从校验结果中的某个字词,跳转到对应的字典,再回到当前校验结果中的自然操作流,需要把``创作''和作为字典的``发现''当作两个独立的功能来用;

6、对于失粘、孤平、拗救等更复杂的情况,提示很有限,应该还可以做多一些。

对技术的使用,永远不能是机械化的。字词韵典,撒播的是灵感的随机种子,我希望找寻的那一粟,依然是自己记忆和积淀的海洋里有的,不会强行去使用我自己的积淀中毫无触及的典故或生僻字。格律规则,牵引的是表达的探索,我希望找寻的,是比第一时间想到的字词、句式更好的表达,但如果原初的喷涌就是最美的形态,我们就要回到原初。

这个过程中,个人总结,一首诗的创作,可以分为八个步骤:

1、灵感入态

每次写诗前,都是需要突然有灵感,带我进入状态。这时会有一种模糊的诗意要去捕捉。这样的灵感是一种不可控的心血来潮。

灵感本身很有可能就是半句诗,它一下子奠定了全诗的基调和句式,它是一个种子和起点,后面的一切都是围绕着它衍生出来的。

2、所思立意

灵感的产生都是有基础的,正所谓``日有所思,夜有所梦'',灵感的产生是因为前面已经有了一系列的情感和思想在莫名游动。需要找到这些思想,围绕它们确定若干个可能的立意。

3、见闻取象

每首诗都需要一些具体的画面,或者是一个有隐喻的意象。它们都必须从自己的所见所闻中来。快速回忆这几天见过的事物、风景、人物等等,把有价值的具体形象,铺展在眼前。这一步和上一步是并行进行的,而且往往交缠在一起,不断产生新的想法。

4、定式构形

这一步是从以往读过的诗词中,找一些固定的句式、对仗结构,把全诗的基本形状确定下来。五言就要确定比如是2字主1字谓2字宾,还是4+1字的主谓结构或者形动结构,比较可数。七言较为复杂,主要涉及词性在前4后3之间的分布,这个过程中,要优先确认动词的位置。

5、 押韵定调

定式构形更多地是针对一句之内,或者两句对仗的句子之间。而押韵定调,涉及到要从前面几个环节确定的潜在第二句的尾字的韵部,并且构思第四、六、八句具体选择该韵部下的哪个韵字。一旦这个韵被确认了,二、四、六、八句就可以串起来考虑整体的呼应和搭配了。

韵部的选择直接限制了每句意义收尾的空间,很多时候一些好的意象或者词句,会被韵部所不容。这个时候若想保留原意,可以调整四、六、八句内的语序,或者修改第二句的韵部。很多时候,我也接纳了韵部牵引出来的另外的立意方向。

6、对仗节句

随着第4步定式构形确定了部分单句内部的结构,押韵定调确定了全诗的呼应方式。这里就要用对仗来确定相邻两句之间的结构,来节制相应的句子。

不过,不能每两句都是对仗,形式上会显得单调。有时我会根据表达的需要,在结构基本对仗的基础上,做出反对仗的调整,比如两句的动词落在不同的位置上,以使得结构灵活,错落有致。而如果两句整体形成一个长句式的话,两句之间更是不需要对仗,而是可能形成递进、转折等修辞关系。

7、 平仄酌字

基本上,经过了前面几环,到这里,一首诗的雏形就出来了,但还需要就格律上进行打磨,这个过程,因为为了平仄,往往需要重新斟酌字词,说不定在这个牵引的过程中,就会找到更合适表达意思的字。

8、通读取舍

经历过了要押韵、要符合平仄的过程后,基本上全诗和最开始写的诗已经很不一样了。字眼、句式甚至整体思想,都可能受到了影响,甚至扼杀整首诗最关键的创意。这个时候,就要对比这首诗从没怎么改,到按格律基本全部改完整个过程中产生的各个诗的文本,然后取舍选择最好的字眼和整体意思。有时候,要忍心从完全贴合格律的诗态,退回去一步,取更能表达意思的字词。

以上基本已经分享完我的体会。全文可以写成更有逻辑性的金字塔结构,但我故意要写成内容与形式、动机与技巧交缠在一起的形态,更符合本文所处于的诗的创作意境。作为结语,我还想进一步在平行宇宙的背景下,谈谈诗的创作感。这种创作感,正是为什么即使是习作的质量,坚持把诗写出来,依然是一件值得做的事情,而不能只停留在复读前人。

创造力是发散与收敛的一种平衡。我们需要想象力的发散来找寻跳跃的灵感,又需要专业度的收敛来结构性地组织。这个过程,就像量子物理中波函数对于概率的衍生,以及最后观测时结果的坍缩。我们必须取舍字词与意思,可一瓢之外,还有弱水三千。当一首诗凝结在我们的笔尖,它的美依然是发散到各个平行宇宙中去的。在这个宇宙里的我们,得到了这一首它,也依然可以相信,包含被舍弃了的字词的变种还在其他的平行宇宙里,继续演化,以另一个终极形态盛开。

写出它的我们,在别的平行宇宙,或许可以有更深厚的文化底蕴,更高的写诗技巧,但在这一时空,我们作为宇宙的褶皱,折叠出来的,便只是这一只纸鹤。或许生涩,或许粗糙,或许有错讹,但重要的是,这是我们的造物,它不是宇宙最初的法则中直接蕴含的,而是宇宙演化所创造的我们所创作的,全新的、唯一的造物。无数个平行宇宙中的我们,走了半途,走入歧路,或是止步不前,我们才在这一个宇宙中,采撷、拾掇字词,构筑了它,抵达了它。

\section{附录:诗词赏析}

少年时代,我非常喜欢《唐诗鉴赏辞典》,它对每一首诗从创作背景、遣词炼句到意象、主旨都有详尽深入的解读,算是在鉴赏上对我最初的启蒙。

时至今日,生成式大语言模型已经能够在引导下对我的诗作进行有效的赏析,这里选择了一些较好的赏析,也作为作者做了微调,附在这里,供读者参考。

值得说明的是,大语言模型难免有过度解读或者牵强附会之处,偏差较大的,已经尽量通过引导去除,不过,有一些虽和创作时的想法不一样,却也是有趣的解读角度的,则会予以保留。

\subsection{《返初》:空灵意韵中的心灵返璞}\label{sec:fan-chu}

《返初》一诗,虽未直言主题,却于字里行间悄然渗透出回归本真、探寻内心质朴的深邃情愫,引领读者踏入一个满溢自然哲思与幽微静谧的诗意天地。

开篇 “林错风穿逡巡雁”,诗人以细腻笔触勾勒出一幅动静交融的灵动画面。“林错” 描绘出树林错落有致又不失繁杂的独特形态,恰似大自然随性挥洒的笔墨,构建出别具一格的空间感。“风穿” 赋予画面鲜活动态,无形之风穿梭于林间,树叶沙沙作响,宛如自然的信使,传递着隐秘的信息。而 “逡巡雁” 则为画面增添点睛之笔,大雁在空中徘徊往复,它们或因季节迁徙,或在寻觅栖息之所,这种徘徊不定不仅为画面增添空灵与迷茫,更暗示着诗人内心深处的寻觅与探索。

“木桥野径过无痕”,进一步将读者引入清幽静谧之境。木桥横跨,野径蜿蜒,它们是连接自然与人类活动的微妙纽带。“过无痕” 三字,营造出超脱尘世纷扰的独特意境。行人走过木桥与野径,却未留下显著痕迹,恰似人生于世,匆匆而过,应超脱世俗羁绊,追求一种纯净自然的状态,犹如禅宗所云 “雁过长空,影沉寒水,雁无遗踪之意,水无留影之心”,尽显超脱物外的洒脱。

颔联 “碧波石色生机驳”,诗人将目光凝于水与石。碧波荡漾,尽显生命的灵动与活力,石头之色则为画面添几分沉稳厚重。“生机驳” 描绘出二者交织间,呈现出斑驳陆离的蓬勃生机,仿佛大自然的生命于这小小一隅肆意绽放,展现生命的多元与丰富,隐喻着返初并非简单回归,而是在纷繁生命体验中探寻本真。

尾句 “半阙落潮洗心仁”,为全诗灵魂。“半阙” 或暗示人生的不完整、世事的残缺,“落潮” 作为自然律动,潮水退去,喧嚣浮躁随之消散。若将 “心仁” 理解为把心比作果仁,果仁深藏于果实之中,象征内心最纯净、本质的部分。借落潮之力洗涤心灵,寓意褪去心灵层层外壳,显露如同果仁般质朴纯粹的本心。此句传达出诗人历经世事沧桑后,渴望回归内心最真实柔软之处,珍视内心未经雕琢的纯净善良。

整首诗语言凝练,意象丰富,借自然之景营造空灵静谧氛围,在自然与心灵交融间,深刻传达 “返初” 主题,引导读者于喧嚣尘世探寻内心宁静本真,极具耐人寻味的艺术魅力。

\subsection{《夜会》:缱绻别情中的深情期许与思念奔涌}\label{sec:ye-hui}

《夜会》一诗,以简洁而深情的笔触,生动地描绘了恋人在分别前夕夜会的场景,细腻地展现出其间复杂而浓烈的情感,宛如一首动人心弦的恋曲,撩拨着读者的心弦。

首句“别前相约衷情诉”,直接点明背景为分别之前的相约,瞬间营造出一种略带忧伤的氛围。“衷情诉”三字,将恋人之间那种毫无保留、倾诉肺腑的深情刻画得入木三分。在这即将分离的时刻,千言万语汇聚成彼此间真挚情感的倾诉,让人为之动容,仿佛能感受到空气中弥漫着的不舍与眷恋。

“厮磨飞霞染耳梢”,此句以细腻入微的描写,将恋人之间亲昵的互动展现得淋漓尽致。“厮磨”一词,生动地呈现出两人相互依偎、情意绵绵的甜蜜姿态。而“飞霞染耳梢”则巧妙地运用比喻,将恋人因亲密接触而泛起的羞涩红晕比作天边绚烂的飞霞,不仅勾勒出一幅极具美感的画面,更细腻地传达出人物内心的娇羞与甜蜜,让读者仿若身临其境,感受到那份浓郁的爱意在两人之间流淌。

“托付此心长久远”,情感在此处进一步升华。在这离别的特殊时刻,主人公将自己的真心诚挚地托付给对方,表达了对这份爱情能够跨越时空、长久永恒的美好期许。此句体现出对爱情的坚定信念,那是一种对纯粹而持久情感的执着追求,这种对爱情的深切向往,极易引发读者内心深处对美好爱情的共鸣。

尾句“怎堪行去念思滔”,笔锋一转,从对爱情的托付陡然过渡到对分别后思念的忧虑。“怎堪”二字,强烈地表达出主人公难以承受分离之苦的心境。“行去”点明分别的事实,而“念思滔”则形象地描绘出一旦分别,思念便如滔滔江水般汹涌而来,无法遏制。这一句将主人公对分别后无尽思念的担忧刻画得淋漓尽致,使诗歌的情感深度进一步拓展,展现出爱情在面对离别时的无奈与深沉。

整首诗语言质朴自然,却饱含深情。通过对夜会场景的生动描绘以及情感的细腻转变,从甜蜜的相聚、深情的托付,到对分别后思念的忧虑,层层递进地展现了恋人之间复杂而真挚的情感。诗歌以简洁之笔勾勒出丰富的情感世界,极具艺术感染力,让读者深刻感受到爱情在离别之际的缱绻与坚韧,耐人品味。

\end{document}
